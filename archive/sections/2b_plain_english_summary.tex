\section{Plain English Summary}

\textbf{Aim}

Our aim is to improve the benefit that can be achieved from the use of the medical stroke treatment (\textit{thrombolysis}) for people who have had a stroke caused by a clot. This treatment is only available for the first few hours of stroke care. Currently there is a lot of variation between each hospital at providing this care. We currently use various modelling techniques to understand the impact of that variation on patients, and the impact on the use of resources in the health care setting. We would like to extend and strengthen our current work by using a number of \textit{causal inference methods}, with the aim to rule out our current observations being due to a coincidence. We would like to identify, with confidence, the cause-and-effect relationships. We will then be able to produce tailored guidance, for each hospital, about how they can improve their patient outcomes. 

We already have very strong direct links for how our findings will benefit patients. NHS-England will use our findings to support specific hospitals with their improvements, and the national stroke audit will incorporate our findings from spring 2024. The techniques and skills developed here will be used immediately in our other stroke care related projects, and will provide a huge benefit to many future projects. 

\textbf{Background to the research}

Stroke is one of the leading causes of death and disability. For every 10 strokes, 8 are caused by a clot in the brain. For these patients there is the possibility to reduce, or remove, the clot by use of thrombolysis (a medical stroke treatment). The use of thrombolysis varies a lot between hospitals, and the overall use is only about half of the NHS target. It has been stuck at this lower rate for nearly 20 years. Resources are currently available to set up new mechanisms to help hospitals improve. This includes the newly established NHS-England \textit{communities of practice of thrombolysis}, which brings together stroke teams with a large difference in their use of thrombolysis. Our modelling work can provide additional insight to this improvement process, by identifying the most effective change for each hospital. 

We have already identified some potential reasons for this variation in stroke care between different hospitals. For example, the largest reason comes from the different processes occurring at each hospital (including each hospital having their own way to decide which patient should receive the medical stroke treatment, \textit{thrombolysis}). Less variation is coming from the difference in the characteristics of the patients attending each hospital. Another useful insight from our modelling work shows that for people suffering a mild stroke, early use of thrombolysis appears to reduce the risk of death, but late use can increase the risk of death. These are important observations that may be used to support a change in clinical practice. However, our current models are not the right type for us to state any \textit{cause-and-effect} conclusions. There is a chance that any current observations are a coincidence. For example, if early use of thrombolysis is linked to a reduction in death, can we be sure this isn't just because the patients receiving treatment earlier have characteristics that make them less likely to die anyway? We wish, for example, to confirm that early use of thrombolysis is actually causing the observed reduction in death. To do this we would use methods known as \textit{causal inference analysis}. By extending our current work to using a number of \textit{causal inference methods}, we will strengthen our findings by ruling out any current observation being due to a coincidence.

\textbf{Design and methods used}

We work together with the Sentinel Stroke National Audit Programme. Our models are based on their detailed stroke dataset, that contains a record for each patient that has had a stroke. When we use data that has been collected as part of the daily hospital routine, no modelling method will be perfect for confirming a cause-and-effect. But by using a number of \textit{causal inference methods} we can test our observations as much as possible. The \textit{causal inference methods} that might be used include \textit{target trial emulation} (where patients are only included if they represent the type that were included in a relevant clinical trial), or \textit{matching} (where matched pairs of patients are found that are very similar apart from the one characteristic that we think may be important). We will hold stakeholder workshops that will help us to develop the work so that it is understandable by, and useful for, stakeholders. We will use our existing stakeholder network across the Sentinel Stroke National Audit, including the newly established NHS-England \textit{communities of practice of thrombolysis}.

\textbf{Patient and public involvement}

We regularly and consistently discuss our work with an engaged group of patients and carers. This process has previously helped us to have a clear way to communicate the models and the results, which in turn has improved our knowledge of the models. The patients and carers have also guided our work to look at what benefits patient outcomes, and not just about meeting arbitrary targets on thrombolysis use. The same process will continue for this project.

\textbf{Dissemination}

Our findings will be shared and used by the national stroke audit and NHS-England. Being published in the monthly stroke audit will also give the clinical stroke community easy access to our findings. We will also publish our work in leading stroke journals, and present it at stroke conferences. All of our detailed work is published online for anyone to freely access and use.