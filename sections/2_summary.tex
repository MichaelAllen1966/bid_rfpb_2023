\section{Summary (long)}
\begin{markdown}

## What is the problem?

Stroke is one of the leading global causes of death and disability. The majority (80\%) of strokes are caused by a clot in the brain (ischaemic stroke). For those patients with ischaemic stroke arriving in the first few hours after stroke onset, there is the possibility to reduce/remove the clot by use of clot-busting medication (thrombolysis) or surgery (thrombectomy). Use of thrombolysis varies very significantly between hospitals (ranging from 5\% to 50\% of patients who arrive in time for treatment. Overall, thrombolysis rates are 11\% of stroke admissions against a NHS target of 20\%. Improving, and reducing the unwanted variation of thrombolysis is a key quality improvement target for NHS England. with the aim of improving net outcomes and reducing unwarrented inequalities of treatment.

## What have we done?

Working with the national stroke audit (SSNAP) and based on SSNAP data for three years of emergency stroke admissions, we have previously built models of the emergency stroke pathway, including using machine learning to compare thrombolysis decisions between hospitals. Recent work has focussed on 1) adding learned outcomes to our machine learning model, and 2) using explainable machine learning methods to make our models more understandable to clinicians (showing why any particular prediction was made). The current methods add significantly to understanding the patterns of variation in clinical decision-making and outcome between hospitals, and helps to build trust in models through making them transparent.

## What gaps still exist?

Though explainable marching learning goes a long way to making helping people understand variation in use of thrombolysis, we do not yet apply any formal causal inference methodology to our modelling. We do not explicitly articulate any assumed/implied causal links. Our models are uncovering relationships between patterns of treatment and outcomes, including reduced/increased risk of death in stroke. As we move towards implementation of our modelling through the national stroke audit and through the NHS-England quality improvement initiative for improving use of thrombolysis, we wish to bolster our modelling by incorporation of more structured caudal inferences analysis.

## What are the aims of the proposed research

This enhanced methodology has potential immediate uses, in coordination with existing NIHR-funded projects, that will benefit patients and the NHS. Though we expect these methods to have wide range and continued application in the team's work, we will develop these methods around central research questions, based on analysing more than 300,000 patient records in SSNAP. These questions align with currently funded work that will inform organisation and delivery of specialist stroke services.

The key aim of the proposed research is to incorporate causal inference methodology into our clinical pathway simulation and machine learning analysis of emergency stoke care. This methodology will be build around Judea Pearl's structured methods of proposing and testing of causal associations. Specific methods will include:

* Articulating proposed causal relationships through use of 'directed acyclic graphs' (DAGs) - these make assumed or hypothesised causal and non-causal relationships clear. They are easily understood by non-technical audiences, and so form an excellent basis for discussions and workshops to explore proposed causal relationships with clinical experts.

* Testing proposed causal relationships with a series of methods. No method alone is perfect, and so we wish to use multiple methods to test proposed causal relationships. These specific methods include:
    
    * Target trial emulation - mimicking a clinical trial from observational data, and then extending to groups that were not included in the original trial.
    
    * Matching - creation of match populations of control and test patients (using nearest-neighbour methods or propensity scoring, so that patients are similar in both groups apart from the feature with proposed causal influence.
    
    * Propensity weighting - weighting the contribution of patients to the final outcome measurement based on their likelihood of not-receiving or receiving treatments. This method gives most weight to patients not treated in the expected way, or patients who are borderline in whether they would receive treatment or not.

    * Stratification of results by feature values or propensity score (likelihood of receiving thrombolysis).
    
    * Covariate adjustment - inclusion of all potential confounding variables in the predictive model.
    
    * Instrument variable analysis - comparing outcomes depending on an 'instrument' that does not directly effect outcome, but effects the proposed causal feature. In our current model we isolate the likelihood that a hospital will give thrombolysis; this may be used as an instrument to test whether there is a relationship between this feature and outcomes.
    
    * Double machine learning - this method uses two or more predictive machine learning models to calculate influences that cannot be directly measured or predicted.

Most of the planned methods are available in the open-source 'DoWhy' Python library developed by Microsoft. This will enable incorporation into our current open-source methodology (we publish all our code and results openly, e.g. see bit.ly/explainable-ml).
 
### Specific questions to be addressed:

* In subgroups where we see significant inter-hospital variation in use of thrombolysis, is there good evidence that the hospital is causing the variation in thrombolysis use in these groups (rather than this effect coinciding with other factors)? Example subgroups are 1) Mild stroke (at presentation), 2) Presence of prior disability, and 3) Imprecisely known stroke onset time.

* In subgroups where we see significant changes in outcome, is there good evidence that the main feature of the subgroup is causing the variation in outcome in these groups (rather than this effect coinciding with other factors)? Example subgroups include early vs. late thrombolysis in mild stroke (where early thrombolysis appears to reduce the odds of death, but late thrombolysis increases the odds of death)?

We see these methods of broader utility to our work across stroke. In a related projected, for example, we would like to know what effect will extending ambulance travel times (in order to take more patients to specialist stroke centres offering both thrombolysis and thrombectomy, an alternative specialist procedure for severe stroke) have on patients who do not receive thrombolysis or thrombectomy? Particular focus will be made on haemorrhagic stroke patients who may be easily be confused with ischaemic stroke patients who would benefit from thrombectomy?

During this project we will hold stakeholder workshops, to obtain feedback on the work. This will use our existing stakeholder network across the Sentinel Stroke National Audit, Integrated Stroke Delivery Networks (ISDN) and Integrated Care Systems (ICS). We will also hold regular meetings with our stroke Patient and Carer Involvement group.

## How will this lead to patient benefit?

* The modelling work currentlky being performed\footnote{https://fundingawards.nihr.ac.uk/award/NIHR134326}) is planned to be incorporated into the national stroke audit from Spring 2024. This will provide teams with realistic target thrombolysis use for their own patient population, and identify which area of the stroke pathway to focus on (pathway speed, ascertainment of stroke onset time, thrombolysis decision making). Additionally, NHS-Elect, working with NHS-England, the SAMueL team, and the national stroke audit, have a NHS improvement target to “Improve access to thrombolysis such that by the end of 2027/28, 20\% of stroke patients will receive thrombolysis treatment”. This collaborative will be working with 6 low-thrombolysing teams in the first instance, before extending work to all emergency stroke teams. As part of the QI work the SAMueL team will be providing modelling on use of thrombolysis, including on variation in decision-making between hospitals. We consider it important that this work should, if at all possible, include strengthened work on causality (the links between changes to pathway and decsion making, through to thrombolysis use, and through to outcome) so that we have greater confidence that changes suggested/made will lead to the expected improvement in thrombolysis use and, most importantly, in better outcomes and not 'just' increased thrombolysis use.

* We are working on stroke projects focussing on the pre-hospital pathway\footnote{https://fundingawards.nihr.ac.uk/award/NIHR202361} and the use of mobile stroke units\footnote{https://fundingawards.nihr.ac.uk/award/NIHR153982 (page due to go live)}. The methodological development above will add value to these, and other similar, projects. For example, it has previously been assumed that outcome from hemorrhagic stroke (the 20\% of strokes that are caused by a bleed rather than a clot) is independent of ambulance travel time, but some recent clinical trials on taking stroke patients further, to a hospital with more capabilities for stokes caused by clots, have suggested this may not be true. Using methods such as clinical trial emulation and our large database of stroke data we will have the potential to enhance our pre-hospital stroke care model to better model outcomes of haemorrhagic strokes with alternative pre-hospital pathways.

### Other benefits

In addition to development of the methodology, this project will be used to develop the skills and capabilities of Kerry Pearn (co-PI) who will lead the project with assistance from Michael Allen (Co-PI). The stroke modelling and data science team is a stable team, funded by research grants and NHS-contracted work, and the techniques and skills developed here will find long-term use and benefit.

\end{markdown}