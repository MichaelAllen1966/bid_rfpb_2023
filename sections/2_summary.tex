\section{Summary (long)}
\begin{markdown}

## What is the problem?

Stroke is one of the leading global causes of death and disability. The majority (80\%) of strokes are caused by a clot in the brain (ischaemic stroke). For those patients with ischaemic stroke arriving in the first few hours after stroke onset, there is the possibility to reduce, or remove, the clot by use of clot-busting medication (thrombolysis) or also by physical removal for severe strokes (thrombectomy). Use of thrombolysis varies very significantly between hospitals, ranging from 5\% to 50\% of patients who arrive in time for treatment. Overall, thrombolysis rates are 11\% of stroke admissions against a NHS target of 20\%. NHS-England have a target of improving use of thrombolysis, and reducing variation between hospitals.

## What have we done?

Working with the Sentinel Stroke National Audit Programme (SSNAP) we have previously built models of the emergency stroke pathway. These models are based on SSNAP data for three years of emergency stroke admissions, and include using machine learning to compare thrombolysis decisions between hospitals. Recent work has focussed on 1) predicting patient disability outcome using machine learning models, and 2) using explainable machine learning methods to make our models more understandable to clinicians (showing why any particular prediction was made). The current methods that we have used add significantly to understanding the patterns of variation in clinical decision-making and outcome between hospitals, and helps to build trust in models by making them transparent. Our modelling has a direct path to implementation by being incorporated into the national stroke audit, and contributing to the monthly auditing output that informs each acute stroke hospital. Our modelling outputs will also inform the NHS-England quality improvement initiative for improving use of thrombolysis.

## What gaps still exist?

Though explainable machine learning goes a long way to helping people understand variation in use of thrombolysis, we do not yet apply any formal causal inference methodology to our modelling. We do not explicitly articulate any assumed, or implied, causal links. Our models are uncovering relationships between treatment decisions and outcomes, including any changes to the risk of death in stroke. As we move towards implementation of our modelling (through the national stroke audit and through the NHS-England quality improvement initiative for improving use of thrombolysis), we wish to significantly enhance the modelling by incorporating more structured causal inferences analysis. The causal inference work will build on the novel work performed so far (e.g. using derived features from the explainable machine learning work as putative *instruments* in instrument variable analysis).

This method development is very likely to be transferable to other work (both within our team, and to other people applying machine learning to the healthcare setting) to identify what causes variation in the services provided by the NHS.

## Data and main outcome measures

The data we use for our work is patient-level data from SSNAP. SSNAP collects data on essentially all emergency stroke admissions across England, Wales and Northern Ireland. The data is a rich collection of information about both the patient and the pathway (and pathway times) that the patient goes through. We are primarily interested in the reperfusion (clot reduction or removal) treatment a patient receives, and the outcome in terms of survival and disability level at discharge. Discharge is measured using the modified Rankin Scale which is a seven-point measure of the ability, or otherwise, to live independently.   

## What are the aims of the proposed research and method development?

This enhanced methodology has potential immediate uses, in coordination with existing NIHR-funded projects, that will benefit patients and the NHS. Though we expect these methods to have a wide range and continued application in the team's work (as well as being transferable to other teams), we will develop these methods around central research questions, based on analysing more than 300,000 patient records in SSNAP. These questions will align with currently funded work that already has an established path for implementation, with the aim to inform organisation and delivery of specialist stroke services.

The key aim of the proposed research is to incorporate causal inference methodology into our clinical pathway simulation and explainable machine learning analysis of emergency stoke care. This methodology will be built largely around Judea Pearl's structured methods of proposing and testing of causal associations. Specific methods will include:

* Articulating proposed causal relationships through use of 'directed acyclic graphs' (DAGs) - these make assumed or hypothesised causal and non-causal relationships clear. They are easily understood by non-technical audiences, and so form an excellent basis for discussions and workshops to explore proposed causal relationships with clinical experts.

* Testing proposed causal relationships with a series of methods. No method alone is perfect, and so we wish to use multiple methods to test proposed causal relationships. These specific methods (in an approximate order of complexity) include:

    * Target trial emulation - mimicking the original thrombolysis clinical trials from observational data, and then extending to groups that were not included in the original trial.

    * Natural experiments - comparing outcomes when changes in services in any region are known to have occurred. This will likely be of most use with wider roll-out of thrombectomy.

    * Covariate adjustment - identify all of the potential confounding variables, and include all of them in a predictive model. This is a slightly different approach to normal predictive machine learnign where feartures are usually chosen by those most contributing to accuracy.

    * Matching - creation of matched populations of control and test patients. Matching attempts to create cohorts of patients that are similar in both groups apart from the feature with proposed causal influence, allowing for an estimation of the effect of the causal feature under study. 

    * Stratification of results by feature values or propensity score, allowing comparisons of similar groups of patients.

    * Instrumental variable analysis - comparing outcomes depending on an 'instrument' that does not directly effect outcome, but effects the proposed causal feature. In our current model we isolate how a hospital affects the likelihood that a patient will be given thrombolysis; we will use this as an instrument to test whether there is a relationship between this feature and outcomes. This will provide a novel link between explainable machine learning methods and causal inference work.
           
    * Inverse propensity weighting - weighting the contribution of patients to the final outcome measurement based on their likelihood of receiving, or not-receiving, thrombolysis. This method gives most weight to patients not treated in the expected way, or to patients who are borderline in whether they would receive treatment or not.   
       
    * Double machine learning - this method uses two machine learning models to calculate influences that cannot be directly measured or predicted using a single machine learning model. This method isolates a features causal relationship with the outcome by fitting the residuals of the feature to the residuals of the outcomes, each as calculated by predicting each from a common set of potential confounding features (thus removing their association).

Add, possibly:

    * Use propensity score in ML outcome models

    * Doubly robust methods - e.g. control for instrumental variable and propensity score.

    * Causal mediation analysis

    * Causal trees

Most of the planned methods are available in the open-source 'DoWhy' Python library developed by Microsoft. This will enable incorporation into our current open-source methodology (we publish all of our code and results openly, e.g. see bit.ly/explainable-ml).
 
### Specific questions to be addressed:

* In subgroups of patients where we see significant inter-hospital variation in use of thrombolysis, is there good evidence that the hospital is causing the variation in thrombolysis use in these groups (rather than this effect coinciding with other factors)? Example patient subgroups are 1) Mild stroke (at presentation), 2) Presence of prior disability, and 3) Imprecisely known stroke onset time.

* In subgroups of patients where we see significant changes in outcome, is there good evidence that the main feature of the patient subgroup is causing the variation in outcome in these groups (rather than this effect coinciding with other factors)? Example patient subgroups include early vs. late thrombolysis in mild stroke (where early thrombolysis appears to reduce the odds of death, but late thrombolysis increases the odds of death)?

We see these methods having a broader application to our other stroke research projects. In a related projected, for example, we would like to know what effect will extending ambulance travel times (in order to take more patients to specialist stroke centres offering both thrombolysis and thrombectomy) have on patients who do not receive thrombolysis or thrombectomy? Particular focus will be made on haemorrhagic stroke patients who, in the pre-hosptial setting, may easily be misclassified as an ischaemic stroke patient who would benefit from thrombectomy.

During this project we will hold stakeholder workshops to obtain feedback on the work. This will use our existing stakeholder network across the SSNAP, Integrated Stroke Delivery Networks (ISDN) and Integrated Care Systems (ICS). We will also hold regular meetings with our stroke Patient and Carer Involvement group.

## How will this lead to patient benefit?

* The modelling work currently being performed\footnote{https://fundingawards.nihr.ac.uk/award/NIHR134326}) is planned to be incorporated into the national stroke audit from Spring 2024. This will provide teams with realistic target thrombolysis use for their own patient population, and identify which area of the stroke pathway to focus on (for example pathway speed, ascertainment of stroke onset time, thrombolysis decision making). Additionally, NHS-Elect (who are working with NHS-England, the SAMueL team, and the national stroke audit) have an NHS improvement target to “Improve access to thrombolysis such that by the end of 2027/28, 20\% of stroke patients will receive thrombolysis treatment”. This collaborative will be working with 6 low-thrombolysing teams in the first instance, before extending work to all emergency stroke teams. As part of the quality improvement work, the SAMueL team will be providing modelling on use of thrombolysis, including on variation in decision-making between hospitals. We consider it important that this work should, if at all possible, include strengthened work on causality (the links between changes to pathway and decision making, through to thrombolysis use, and through to outcome) so that we have greater confidence that changes suggested (and potentially made) will lead to the expected improvement in thrombolysis use and, most importantly, in better outcomes and not 'just' increased thrombolysis use.

* We are working on stroke projects focussing on the pre-hospital pathway\footnote{https://fundingawards.nihr.ac.uk/award/NIHR202361} and the use of mobile stroke units\footnote{https://fundingawards.nihr.ac.uk/award/NIHR153982}. The methodological development above will add value to both of these, and also to other similar, projects. For example, it has previously been assumed that outcome from hemorrhagic stroke (the 20\% of strokes that are caused by a bleed rather than a clot) is independent of ambulance travel time, but some recent clinical trials on taking stroke patients further, to a hospital with more capabilities for stokes caused by clots, have suggested this may not be true. Using methods such as clinical trial emulation and our large database of stroke data we will have the potential to enhance our pre-hospital stroke care model to better model outcomes of haemorrhagic strokes with alternative pre-hospital pathways.

### Other benefits

In addition to development of the methodology, this project will be used to develop the skills and capabilities of Kerry Pearn (Co-PI) who will lead the project with assistance from Michael Allen (Co-PI). The stroke modelling and data science team is a stable team, funded by research grants and NHS-contracted work, and the techniques and skills developed here will find long-term use and benefit.

\end{markdown}