\section*{Plain English Summary}

The focus of our team's work is on using explainable machine learning, clinical pathway simulation, and geographic modelling, applied to national clinical audit data to identify between-hospital variation in clinical decision-making and processes, and understanding the impact of that variation on patients and on the use of health service resources. Our work is in collaboration with the Sentinel National Stroke Audit Programme, and focuses mostly on the emergency stroke pathway, but is applicable across other clinical areas.

In our work we uncover potential causes of between-hospital variation. For example, in the use of clot-busting medication (\textit{thrombolysis}) in stroke, the majority of the large between-hospital variation in use of these drugs appears to come from differences in decision-making and in-hospital processes, rather than from differences in the characteristics of the patients attending each hospital. Another example is that appears early use of these clot-busting drugs reduces the risk of death even in mild strokes, but late use can increase the risk of death. These are important observations that may be used to advocate for change in clinical practice. Though our models attempt to isolate the effects of different factors, we would like to use multiple different methods to strengthen the \textit{cause-and-effect} conclusions from our current models. This will help rule out our current observations being coincidence (for example, if early use of clot-busting drugs is linked to a reduction in death, can we be sure this isn't just because the patients receiving treatment earlier have characteristics that make them less likely to die anyway?). We wish, for example, to confirm that early use of thrombolysis is actually causing the observed reduction in death. To do this we would use methods known as \textit{causal inference analysis}.

When using data that has been collected in routine clinical practice, no method is perfect for confirming a cause-and-effect, but by incorporating multiple methods into our analysis we wish to test our observations from our machine learning as much as possible. Example methods that might be used include \textit{target trial emulation} (where the patient set is restricted to those that have been through clinical trials in a similar area of study), or \textit{matching} where matched pairs of patients are found that are very similar apart from the one characteristic that we think may be important.

Adding methods that more directly investigate cause-and-effect will substantially strengthen our suite of tools. We have recently added \textit{explainability} to our machine-learning models (so that we can see what led to any particular prediction the model made). Adding stronger cause-and-effect analysis would help test our models further and build more trust in conclusions and recommendations made. 

During this project we will hold stakeholder workshops, to obtain feedback on the work. This will use our existing stakeholder network across the Sentinel Stroke National Audit including the newly established NHS-England \textit{communities of practice of thrombolysis}, which bring together stroke teams with differing use of clot-busting drugs. We will also regularly involve patients and carers in discussing the work - our experience is that this very significantly helps test our knowledge on the models, and helps us to develop much clearer communication of the models and the results.

In addition to development of the methodology, this project will be used to develop the skills and capabilities of Kerry Pearn (co-PI) who will lead the project with assistance from Michael Allen (Co-PI). The stroke modelling and data science team is a stable team, funded by research grants and NHS-contracted work, and the techniques and skills developed here will find long-term use and benefit.