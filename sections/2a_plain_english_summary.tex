\section*{Plain English Summary}

We are a group of researchers who use various modelling techniques to identify the variation that occurs across different hospitals during the first few hours of stroke care. We use these models to understand the impact of that variation on patients, and the impact on the use of resources in the health care setting. We work together with the Sentinel Stroke National Audit Programme. Our models are based on their detailed stroke dataset, that contains a record for each patient that has had a stroke.

We have identified some potential reasons for this variation in stroke care between different hospitals. For example, the largest reason comes from the different processes occurring at each hospital (including each hospitals having their own way to decide which patient should receive the medical stroke treatment, (\textit{thrombolysis}). Less variation is coming from the difference in the characteristics of the patients attending each hospital. Another useful insight from our modelling work shows that for people suffering a mild strokes, early use of thrombolysis appears to reduce the risk of death, but late use can increase the risk of death. These are important observations that may be used to support a change in clinical practice. However, our current models are not the right type for us to state any \textit{cause-and-effect} conclusions. There is a chance that any current observations are a coincidence. For example, if early use of thrombolysis is linked to a reduction in death, can we be sure this isn't just because the patients receiving treatment earlier have characteristics that make them less likely to die anyway?). We wish, for example, to confirm that early use of thrombolysis is actually causing the observed reduction in death. To do this we would use methods known as \textit{causal inference analysis}. We would like to extend and strengthen our current work by using multiple different \textit{causal inference methods}, with the aim to rule out our current observations being due to a coincidence.

When we use data that has been collected as part of the daily hospital routine, no modelling method will be perfect for confirming a cause-and-effect. But by using multiple \textit{causal inference methods} we can test our observations as much as possible. The \textit{causal inference methods} that might be used include \textit{target trial emulation} (where the patient set is restricted to those that have been through clinical trials in a similar area of study), or \textit{matching} (where matched pairs of patients are found that are very similar apart from the one characteristic that we think may be important).

Adding methods that investigate cause-and-effect will strengthen our work. We have recently added \textit{explainability} to our predictive models. This allows us to see how the model made each prediction. Adding stronger cause-and-effect analysis would help to test our models even further, and build more trust in the results and recommendations. 

During the project we will hold stakeholder workshops that will help us to develop the work so that it is understable by, and useful for, stakeholders. We will use our existing stakeholder network across the Sentinel Stroke National Audit, including the newly established NHS-England \textit{communities of practice of thrombolysis} (this brings together stroke teams with a large difference in their use of thrombolysis). We will also regularly discuss our work with patients and carers. This process has previously helped us to have a clear way to communicate the models and the results, which in turn has improved our knowledge of the models.

This work will also be immediately used in the other projects that we undertake with the national stroke audit, and with national and regional NHS organisations. 

In addition to development of the methodology, this project will be used to develop the skills and capabilities of Kerry Pearn (Co-PI) who will lead the project with assistance from Michael Allen (Co-PI). The stroke modelling and data science team is a stable team, funded by research grants and NHS-contracted work. The techniques and skills developed here will be used and provide a huge benefit many future projects.