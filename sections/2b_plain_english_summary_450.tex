\section*{Plain English Summary}

\textbf{Aim}

Our aim is to improve the benefit that can be achieved from the use of the medical stroke treatment (\textit{thrombolysis}) for people who have had a stroke caused by a clot. This treatment is only available for the first few hours of stroke care. There is a large variation between each hospital at providing this care. We currently use various modelling techniques to understand the impact of this variation on patients, and on the use of health care resources. We would like to extend and strengthen our work by using \textit{causal inference methods}, with the aim to rule out observations being due to a coincidence, and identify the cause-and-effect relationships. We will produce guidance for each hospital about how they can improve their patient outcomes. 

We already have strong direct links for how our findings will benefit patients. NHS-England will use our findings to support specific hospitals with their improvements, and the national stroke audit will incorporate our findings from spring 2024. The knowledge developed here will be used immediately in our other stroke care related projects, and will provide a huge benefit to many future projects. 

\textbf{Background to the research}

Stroke is one of the leading causes of death and disability. For every 10 strokes, 8 are caused by a clot in the brain. For these patients there is the possibility to reduce, or remove, the clot by use of thrombolysis. Thrombolysis use varies a lot between hospitals, and the overall use is only about half of the NHS target. Resources are currently available to create new mechanisms to help hospitals improve. This includes the newly established NHS-England \textit{communities of practice of thrombolysis}, which brings together stroke teams with a large difference in their thrombolysis use. Our modelling work can provide additional insight to this improvement process, by identifying the most effective change for each hospital.

Our models have already identified some potential reasons for this variation in stroke care between hospitals. One useful insight shows that for people suffering a mild stroke, early use of thrombolysis appears to reduce the risk of death, but late use can increase the risk of death. This important observation may be used to support a change in clinical practice. However, our current models are not the right type for us to state any \textit{cause-and-effect} conclusions. To be able to confirm that early use of thrombolysis is actually causing the observed reduction in death we would use \textit{causal inference analysis}. By extending our current work to using \textit{causal inference methods}, we will strengthen our findings by ruling out any current observation being due to a coincidence.

\textbf{Design and methods used}

Our models are based on the Sentinel Stroke National Audit Programme detailed stroke dataset. It contains a record for each patient that has had a stroke. No modelling method will be perfect for confirming a cause-and-effect relationship from routinely collected data. But by using a number of \textit{causal inference methods} we can best test our observations. The \textit{causal inference methods} that might be used include \textit{target trial emulation} (only included patients if they represent those that were included in a relevant clinical trial), or \textit{matching} (matched pairs of patients are included that are very similar apart from the one characteristic that may be important). We will hold stakeholder workshops to help us develop the work so that it is understandable by, and useful for, stakeholders. We will use our existing stakeholder network across the Sentinel Stroke National Audit, including the NHS-England \textit{communities of practice of thrombolysis}.

\textbf{Patient and public involvement}

We regularly discuss our work with an engaged group of patients and carers. This process has previously helped us to communicate the models and the results clearly, which in turn has improved our knowledge of the models. The patients and carers also guided our work to look at what benefits patient outcomes, and not just about meeting arbitrary targets. The same process will continue for this project.

\textbf{Dissemination}

Our findings will be shared and used by the national stroke audit and NHS-England. Being published in the monthly stroke audit will also give the clinical stroke community easy access to our findings. We will also publish our work in leading stroke journals, and present it at stroke conferences. All of our detailed work is published online for anyone to freely access and use.